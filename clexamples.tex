% 编译方式: xelatex -> bibtex -> xelatex*2
\documentclass{ctexart}
\usepackage{amsmath}
\usepackage{amssymb}
\usepackage{graphicx}
\usepackage{hyperref}
\usepackage{syntonly}
\usepackage{textcomp}
\usepackage{verbatim}
% \syntaxonly
\hyphenation{}
\hypersetup{colorlinks}
\title{\LaTeX{}范例集}
\author{GasinAn}
\begin{document}
    \maketitle
    这是一份\LaTeX{}基础范例集!
    \section{基础}
    \subsection{段落}
    这话儿在第一段.
    这话儿还是在第一段.

    这话儿才是在第二段!
    \subsection{特殊字符}
    \~ \# \$ \% \^ \& \_ \{ \} \\ % 这是错的!
    \~{} \# \$ \% \^{} \& \_ \{ \} \\ \textbackslash % 这是对的!
    \section{排版}
    \subsection{强制分行}
    这话儿在第一行.\newline
    这话儿在第二行.
    \subsection{强制分页}
    这话儿如果在第一页.\newpage
    这话儿就一定在第二页.
    \subsection{引号}
    '错', "错".

    `对', ``对''.

    ‘对’,“对”。
    \subsection{横线}
    X-ray

    0--65535页

    Yes---or no?

    是——还是不是?

    $-2147483648$
    \subsection{波浪号}
    这样不好.\~{}

    这样好!$\sim$
    \subsection{角度符号}
    $361^{\circ}$

    $-273.15^{\circ}\mathrm{C}$ (这样更好: $-273.15\,^{\circ}\mathrm{C}$)

    $-273.15$ \textcelsius{} = $-459.67$ \textdegree{}F
    \subsection{省略号}
    这样不好...

    这样好!\dots

    这样也好!\ldots

    这样也好!……
    \subsection{交叉引用}\label{ref}
    像这样引用本小节: ``见第 \pageref{ref} 页的第 \ref{ref} 小节. ''
    \subsection{脚注}
    给整句话或话的一部分加的脚注要放在句号后.\footnote{这是个可爱的小脚注.}
    \subsection{强调}
    强调\emph{强调}强调

    \textit{强调\emph{强调}强调}
    \subsection{无序列表(itemize)}
    \begin{itemize}
        \item 约翰$\cdot$希金斯
        \item 马克$\cdot$威廉姆斯
        \item 罗尼$\cdot$奥沙利文
    \end{itemize}
    \subsection{有序列表(enumerate)}
    \begin{enumerate}
        \item 世锦赛
        \item 英锦赛
        \item 大师赛
    \end{enumerate}
    \subsection{描述列表(description)}
    \begin{description}
        \item[蹬杆] 指以较大的力量击打母球中部, 使得母球在接触目标球之前保持无滚动的状态. 效果是在母球与目标球接触后, 母球与目标球的运动方向的夹角接近90\textdegree. 如果与母球与目标球正碰撞, 理想状态下母球会瞬间静止. 也叫斯登或司登, 英文名为stun. 
        \item[推杆] 指以较小的力量击打母球中部, 使得母球在接触目标球之前因台面的摩擦作用而最终滚向目标球. 效果是在母球与目标球接触后, 母球与目标球的运动方向的夹角小于90\textdegree. 如果与母球与目标球正碰撞, 母球之后会向前运动一段距离. 
    \end{description}
    \subsection{摘录(quote)}
    \begin{quote}
        \dots lensed FRBs, as a powerful probe and completely independent dataset based on a different physical phenomenon, would provide complementary information and therefore are of vital importance to clarify the tension between the latest Planck-inferred $H_0$ and the one from direct local distance ladder observations.\footnote{
            Li, ZX., Gao, H., Ding, XH. \textit{et al}. Strongly lensed
            repeating fast radio bursts as precision probes of the universe.
            \textit{Nat Commun} \textbf{9}, 3833 (2018).}
    \end{quote}
    \subsection{摘要(abstract)}
    \begin{abstract}
        Fast radio bursts (FRBs) are millisecond-duration radio transients of
        unknown physical origin observed at extragalactic distances. It has
        long been speculated that magnetars are the engine powering repeating
        bursts from FRB sources, but no convincing evidence has been collected
        so far. Recently, the Galactic magnetar SRG 1935+2154 entered an active
        phase by emitting intense soft $\gamma$-ray bursts. One FRB-like event
        with two peaks (FRB 200428) and a luminosity slightly lower than the
        faintest extragalactic FRBs was detected from the source, in
        association with a soft $\gamma$-ray/hard-X-ray flare. Here we report
        an eight-hour targeted radio observational campaign comprising four
        sessions and assisted by multi-wavelength (optical and hard-X-ray)
        data. During the third session, 29 soft-$\gamma$-ray repeater (SGR)
        bursts were detected in $\gamma$-ray energies. Throughout the observing
        period, we detected no single dispersed pulsed emission coincident with
        the arrivals of SGR bursts, but unfortunately we were not observing
        when the FRB was detected. The non-detection places a fluence upper
        limit that is eight orders of magnitude lower than the fluence of FRB
        200428. Our results suggest that FRB–SGR burst associations are rare.
        FRBs may be highly relativistic and geometrically beamed, or FRB-like
        events associated with SGR bursts may have narrow spectra and
        characteristic frequencies outside the observed band. It is also
        possible that the physical conditions required to achieve coherent
        radiation in SGR bursts are difficult to satisfy, and that only under
        extreme conditions could an FRB be associated with an SGR burst.\footnote{
            Lin, L., Zhang, C.F., Wang, P. \textit{et al}.
            No pulsed radio emission during a bursting phase of a Galactic magnetar.
            \textit{Nature} \textbf{587}, 63–65 (2020).}
    \end{abstract}
    \subsection{原文打印}
\begin{verbatim}
program hellolatex
print *, "Hello, LaTeX!"
end program hellolatex
\end{verbatim}
\begin{verbatim*}
      PROGRAM HELLOLATEX
      PRINT *, 'HELLO, LATEX.'
      END PROGRAM HELLOLATEX
\end{verbatim*}
    最后一行可以改成 \verb|end| (或 \verb*|      END|).
    \subsection{表格}
    \begin{table}[htbp]
        \centering
        \def\sg{施特菲$\cdot$格拉芙} % 定义新命令!
        \begin{tabular}{c|ccccc}
            \hline
            运动员 & 澳网 & 法网 & 温网 & 美网 & 奥运会\\
            \hline
            \sg & 1988 & 1988 & 1988 & 1988 & 1988 \\
            \hline
        \end{tabular}
        \caption{年度金满贯}
        \label{table} % Put \label here!
    \end{table}
    \subsection{图片}
    \begin{figure}[htbp]
        \centering
        \includegraphics[width=0.8\textwidth]{img/fn.jpeg}
        \caption{费德勒(左)与纳达尔(右)}
        \label{figure} % Put \label here!
    \end{figure}
    \section{数学公式}
    \subsection{公式}
    把行内公式 $E=mc^1$ 写成独立公式:
    \begin{equation*}
        E=mc^1.
    \end{equation*}

    把行内公式 $E=mc^2$ 写成有编号的独立公式:
    \begin{equation}
        E=mc^2. \label{eq2}
    \end{equation}

    把行内公式 $E=mc^3$ 写成有特殊编号的独立公式:
    \begin{equation}
        E=mc^3. \tag{$\ast$} \label{eq3}
    \end{equation}

    式 \eqref{eq2} 显然是对的. 式 \eqref{eq3} 连量纲都不对\dots
    \subsection{文字}
    这样写是错的:
    \begin{equation*}
        x > 0    任意 x \in R_+.
    \end{equation*}

    这样写才是对的:
    \begin{equation*}
        x > 0 \qquad \text{任意} \; x \in \mathbb{R_+}.
    \end{equation*}
    \subsection{上下标}
    \begin{equation*}
        \sum_{i=1}^{100} i = \sum^{100}_{j=1} j.
    \end{equation*}
    \begin{equation*}
        a^x+y \neq a^{x+y}.
    \end{equation*}
    \begin{equation*}
        e^{x^2} \neq {e^x}^2.
    \end{equation*}
    \subsection{根号}
    \begin{equation*}
        \sqrt{5} > \sqrt[5]{5}.
    \end{equation*}
    \subsection{函数名}
    这样写是错的:
    $ 2 \text{lim}_{x \rightarrow \infty} \text{arctan} x = \pi $.

    这样写才是对的:
    $ 2 \lim_{x \rightarrow \infty} \arctan x = \pi $.
    \subsection{模函数}
    \begin{equation*}
        5 \bmod 2 = 1.
    \end{equation*}
    \begin{equation*}
        5 \equiv 1 \pmod{2}.
    \end{equation*}
    \subsection{分数}
    \begin{equation*}
        \frac{\mathrm{d}^{n}y}{\mathrm{d}x^{n}}.
    \end{equation*}
    \begin{equation*}
        \frac{\partial^{2}z}{\partial{x}\partial{y}}.
    \end{equation*}
    \subsection{二项式系数}
    \begin{equation*}
        \binom{n}{k} = \binom{n-1}{k} + \binom{n-1}{k-1}.
    \end{equation*}
    \subsection{符号堆叠}
    $ 2\text{K}\text{Mn}\text{O}_4 \stackrel{\triangle}{\longrightarrow}
      \text{K}_2\text{Mn}\text{O}_4+\text{Mn}\text{O}_2+\text{O}_2\uparrow $,
    $ 2\text{H}_2\text{O}_2 \xrightarrow[]{\text{Mn}\text{O}_2}
    2\text{H}_2\text{O}+\text{O}_2\uparrow $.

    第二个更好!

    数学比化学复杂多了:
    \begin{equation*}
        \sum_{\substack{1\le{i}\le{n}\\j>i}} l_{ij}^2 = 0.
    \end{equation*}
    \subsection{矩阵}
    \begin{equation*}
        \boldsymbol{I} = 
        \begin{bmatrix}
            1 & 0 \\
            0 & 1
        \end{bmatrix}.
    \end{equation*}
    \subsection{分段函数}
    \begin{equation*}
        |x| = 
        \begin{cases}
            -x & x < 0, \\
             0 & x = 0, \\
             x & x > 0. \\
        \end{cases}
    \end{equation*}
    \subsection{连等式}
    \begin{align}
        \nabla\frac{1}{r} & = \nabla\frac{1}{\sqrt{x^2+y^2+z^2}} \label{oeq}\\
            & = -\frac{x\boldsymbol{i}+y\boldsymbol{j}+z\boldsymbol{k}}
                      {\sqrt{x^2+y^2+z^2}^3} \nonumber \\
            & = -\frac{\hat{\boldsymbol{r}}}{r^2}. \tag{$\star$} \label{beq}
    \end{align}

    \eqref{oeq} 能直接从定义得出. 最后的结果 \eqref{beq} 非常漂亮!
    \subsection{空格}
    $ \text{这是} \  \text{1 space.} $

    $ \text{这是} \! \text{-3/18 em.} $

    $ \text{这是} \, \text{3/18 em.} $

    $ \text{这是} \: \text{4/18 em.} $

    $ \text{这是} \; \text{5/18 em.} $

    $ \text{这是} \quad \text{1 em.} $

    $ \text{这是} \qquad \text{2 em.} $
    \subsection{空位}
    这样写是错的: ${}^{99}_{ 9}\text{C}R_{abc}^{   d}$.

    这样写还是错的: ${}^{99}_{\ 9}\text{C}R_{abc}^{\ \ \ d}$.

    这样写才是对的: ${}^{99}_{\phantom{9}9}\text{C}R_{abc}^{\phantom{abc}d}$.
    \newpage
    \subsection{命令小全}
    \begin{table}[!htbp]
        \centering
        \begin{tabular}{cccccc}
            $\dot{a}$ & \verb|\dot{a}| &
            $\bar{a}$ & \verb|\bar{a}| &
            $\hat{a}$ & \verb|\hat{a}| \\
            $\ddot{a}$ & \verb|\ddot{a}| &
            $\vec{a}$ & \verb|\vec{a}| \\
        \end{tabular}
    \end{table}
    \begin{table}[!htbp]
        \centering
        \begin{tabular}{cccccc}
            $<$ & \verb|<| &
            $>$ & \verb|>| &
            $=$ & \verb|=| \\
            $\leq$ & \verb|\leq| &
            $\geq$ & \verb|\geq| &
            $\equiv$ & \verb|\equiv| \\
            $\ll$ & \verb|\ll| &
            $\gg$ & \verb|\gg| \\
            $\in$ & \verb|\in| &
            $\ni$ & \verb|\ni| &
            $\sim$ & \verb|\sim| \\
            $\subset$ & \verb|\subset| &
            $\supset$ & \verb|\supset| &
            $\simeq$ & \verb|\simeq| \\
            $\subseteq$ & \verb|\subseteq| &
            $\supseteq$ & \verb|\supseteq| &
            $\approx$ & \verb|\approx| \\
            $:$ & \verb|:| &
            $\mid$ & \verb|\mid| &
            $\propto$ & \verb|\propto| \\
            $\perp$ & \verb|\perp| &
            $\parallel$ & \verb|\parallel| \\
            $\not\in$ & \verb|\not\in| &
            $\not\ni$ & \verb|\not\ni| &
            $\neq$ & \verb|\neq| \\
            $\not\subset$ & \verb|\not\subset| &
            $\not\supset$ & \verb|\not\supset| &
            $\not\equiv$ & \verb|\not\equiv| \\
            $\not\subseteq$ & \verb|\not\subseteq| &
            $\not\supseteq$ & \verb|\not\supseteq| \\
        \end{tabular}
    \end{table}
    \begin{table}[!htbp]
        \centering
        \begin{tabular}{cccccccc}
            $+$ & \verb|+| &
            $-$ & \verb|-| &
            $\times$ & \verb|\times| &
            $\div$ & \verb|\div| \\
            $\pm$ & \verb|\pm| &
            $\mp$ & \verb|\mp| &
            $\cdot$ & \verb|\cdot| &
            $/$ & \verb|/| \\
            $\cup$ & \verb|\cup| &
            $\cap$ & \verb|\cap| &
            $\oplus$ & \verb|\oplus| &
            $\otimes$ & \verb|\otimes| \\
            $\star$ & \verb|\star| &
            $\ast$ & \verb|\ast| &
            $\dagger$ & \verb|\dagger| &
            $\ddagger$ & \verb|\ddagger| \\
        \end{tabular}
    \end{table}
    \begin{table}[!htbp]
        \centering
        \begin{tabular}{cccccccc}
            $\sum$ & \verb|\sum| &
            $\prod$ & \verb|\prod| &
            $\bigcup$ & \verb|\bigcup| &
            $\bigcap$ & \verb|\bigcap| \\
            $\int$ & \verb|\int| &
            $\oint$ & \verb|\oint| &
            $\bigoplus$ & \verb|\bigoplus| &
            $\bigotimes$ & \verb|\bigotimes| \\
        \end{tabular}
    \end{table}
    \begin{table}[!htbp]
        \centering
        \begin{tabular}{cccc}
            $\leftarrow$ & \verb|\leftarrow| &
            $\longleftarrow$ & \verb|\longleftarrow| \\
            $\rightarrow$ & \verb|\rightarrow| &
            $\longrightarrow$ & \verb|\longrightarrow| \\
            $\leftrightarrow$ & \verb|\leftrightarrow| &
            $\longleftrightarrow$ & \verb|\longleftrightarrow| \\
            $\Leftarrow$ & \verb|\Leftarrow| &
            $\Longleftarrow$ & \verb|\Longleftarrow| \\
            $\Rightarrow$ & \verb|\Rightarrow| &
            $\Longrightarrow$ & \verb|\Longrightarrow| \\
            $\Leftrightarrow$ & \verb|\Leftrightarrow| &
            $\Longleftrightarrow$ & \verb|\Longleftrightarrow| \\
            $\mapsto$ & \verb|\mapsto| &
            $\longmapsto$ & \verb|\longmapsto| \\
            $\uparrow$ & \verb|\uparrow| &
            $\downarrow$ & \verb|\downarrow| \\
        \end{tabular}
    \end{table}
    \begin{table}[!htbp]
        \centering
        \begin{tabular}{cccc}
            $\overleftarrow{AB}$ & \verb|\overleftarrow{AB}| &
            $\underleftarrow{AB}$ & \verb|\underleftarrow{AB}| \\
            $\overrightarrow{AB}$ & \verb|\overrightarrow{AB}| &
            $\underrightarrow{AB}$ & \verb|\underrightarrow{AB}| \\
            $\overleftrightarrow{AB}$ & \verb|\overleftrightarrow{AB}| &
            $\underleftrightarrow{AB}$ & \verb|\underleftrightarrow{AB}| \\
        \end{tabular}
    \end{table}
    \begin{table}[!htbp]
        \centering
        \begin{tabular}{cccc}
            $(a)$ & \verb|(a)| &
            $[a]$ & \verb|[a]| \\
            $\{a\}$ & \verb|\{a\}| &
            $\langle{a}\rangle$ & \verb|\langle{a}\rangle| \\
            $\lfloor{a}\rfloor$ & \verb|\lfloor{a}\rfloor| &
            $\lceil{a}\rceil$ & \verb|\lceil{a}\rceil| \\
            $\lvert{a}\rvert$ & \verb|\lvert{a}\rvert| &
            $\lVert{a}\rVert$ & \verb|\lVert{a}\rVert| \\
        \end{tabular}
    \end{table}
    \begin{table}[!htbp]
        \centering
        \begin{tabular}{cccccccc}
            $\ldots$ & \verb|\ldots| &
            $\cdots$ & \verb|\cdots| &
            $\vdots$ & \verb|\vdots| &
            $\ddots$ & \verb|\ddots| \\
            $\because$ & \verb|\because| &
            $\therefore$ & \verb|\therefore| &
            $\infty$ & \verb|\infty| &
            $\%$ & \verb|\%| \\
            $\nabla$ & \verb|\nebla| &
            $\angle$ & \verb|\angle| &
            $\square$ & \verb|\square| &
            $\varnothing$ & \verb|\varnothing| \\
            $\hbar$ & \verb|\hbar| &
            $\ell$ & \verb|\ell| &
            $\Re$ & \verb|\Re| &
            $\Im$ & \verb|\Im| \\
            $\forall$ & \verb|\forall| &
            $\exists$ & \verb|\exists| &
            $\aleph$ & \verb|\aleph| &
            $\partial$ & \verb|\partial| \\
            $\S$ & \verb|\S| &
            $\P$ & \verb|\P| \\
        \end{tabular}
    \end{table}
    \newpage
    \section{参考文献}
    FRBs are millisecond-duration bright radio transients
    \cite{Li2018,Lin2020}.
    \bibliographystyle{abbrv}
    \bibliography{astro}
    \section{自定义命令}
    \newcommand{\isgood}[3][是好的]{\href{#3}{#2} #1!}

    \isgood{lshort}{https://www.ctan.org/pkg/lshort}

    \isgood[也是好的]{lnotes}{https://github.com/huangxg/lnotes}
    \section{超链接}
    想让\href{clexample.pdf}{《\LaTeX{}范例集》}变得更好?

    可以在~\href{https://github.com/GasinAn/lexamples}{Github}~上
    提交pull request或issue.

    也可以发邮件至~\href{mailto:Gasin185@163.com}{Gasin185@163.com}.
\end{document}